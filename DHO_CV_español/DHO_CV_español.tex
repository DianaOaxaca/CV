%!TEX TS-program = xelatex
%!TEX encoding = UTF-8 Unicode
% Awesome CV LaTeX Template for CV/Resume
%
% This template has been downloaded from:
% https://github.com/posquit0/Awesome-CV
%
% Author:
% Claud D. Park <posquit0.bj@gmail.com>
% http://www.posquit0.com
%
%
% Adapted to be an Rmarkdown template by Mitchell O'Hara-Wild
% 23 November 2018
%
% Template license:
% CC BY-SA 4.0 (https://creativecommons.org/licenses/by-sa/4.0/)
%
%-------------------------------------------------------------------------------
% CONFIGURATIONS
%-------------------------------------------------------------------------------
% A4 paper size by default, use 'letterpaper' for US letter
\documentclass[11pt,a4paper,]{awesome-cv}

% Configure page margins with geometry
\usepackage{geometry}
\geometry{left=1.4cm, top=.8cm, right=1.4cm, bottom=1.8cm, footskip=.5cm}


% Specify the location of the included fonts
\fontdir[fonts/]

% Color for highlights
% Awesome Colors: awesome-emerald, awesome-skyblue, awesome-red, awesome-pink, awesome-orange
%                 awesome-nephritis, awesome-concrete, awesome-darknight

\definecolor{awesome}{HTML}{414141}

% Colors for text
% Uncomment if you would like to specify your own color
% \definecolor{darktext}{HTML}{414141}
% \definecolor{text}{HTML}{333333}
% \definecolor{graytext}{HTML}{5D5D5D}
% \definecolor{lighttext}{HTML}{999999}

% Set false if you don't want to highlight section with awesome color
\setbool{acvSectionColorHighlight}{true}

% If you would like to change the social information separator from a pipe (|) to something else
\renewcommand{\acvHeaderSocialSep}{\quad\textbar\quad}

\def\endfirstpage{\newpage}

%-------------------------------------------------------------------------------
%	PERSONAL INFORMATION
%	Comment any of the lines below if they are not required
%-------------------------------------------------------------------------------
% Available options: circle|rectangle,edge/noedge,left/right

\photo{img/Diana.jpeg}
\name{Diana Hernández Oaxaca}{}

\address{Centro de Ciencias Genómicas, UNAM}

\email{\href{mailto:hoaxaca@ccg.unam.mx}{\nolinkurl{hoaxaca@ccg.unam.mx}}}
\orcid{0000-0001-6218-8297}
\github{DianaOaxaca}

% \gitlab{gitlab-id}
% \stackoverflow{SO-id}{SO-name}
% \skype{skype-id}
% \reddit{reddit-id}

\quote{Soy estudiante de doctorado, próxima a titularme. Estudio la
simbiosis especialista y generalista entre bacterias y sus hospederos:
plantas o animales. Me apasiona estudiar microbiología, bioquímica y
ciencias ómicas. Disfruto de aprender y enseñar, por eso participo
activamente tomando, dando y apoyando en cursos y asesorías}

\usepackage{booktabs}

\providecommand{\tightlist}{%
	\setlength{\itemsep}{0pt}\setlength{\parskip}{0pt}}

%------------------------------------------------------------------------------



% Pandoc CSL macros
\newlength{\cslhangindent}
\setlength{\cslhangindent}{1.5em}
\newlength{\csllabelwidth}
\setlength{\csllabelwidth}{2em}
\newenvironment{CSLReferences}[2] % #1 hanging-ident, #2 entry spacing
 {% don't indent paragraphs
  \setlength{\parindent}{0pt}
  % turn on hanging indent if param 1 is 1
  \ifodd #1 \everypar{\setlength{\hangindent}{\cslhangindent}}\ignorespaces\fi
  % set entry spacing
  \ifnum #2 > 0
  \setlength{\parskip}{#2\baselineskip}
  \fi
 }%
 {}
\usepackage{calc}
\newcommand{\CSLBlock}[1]{#1\hfill\break}
\newcommand{\CSLLeftMargin}[1]{\parbox[t]{\csllabelwidth}{\honortitlestyle{#1}}}
\newcommand{\CSLRightInline}[1]{\parbox[t]{\linewidth - \csllabelwidth}{\honordatestyle{#1}}}
\newcommand{\CSLIndent}[1]{\hspace{\cslhangindent}#1}

\begin{document}

% Print the header with above personal informations
% Give optional argument to change alignment(C: center, L: left, R: right)
\makecvheader

% Print the footer with 3 arguments(<left>, <center>, <right>)
% Leave any of these blank if they are not needed
% 2019-02-14 Chris Umphlett - add flexibility to the document name in footer, rather than have it be static Curriculum Vitae
\makecvfooter
  {November 2023}
    {Diana Hernández Oaxaca~~~·~~~Curriculum Vitae}
  {\thepage~ of \pageref{LastPage}~}


%-------------------------------------------------------------------------------
%	CV/RESUME CONTENT
%	Each section is imported separately, open each file in turn to modify content
%------------------------------------------------------------------------------



\hypertarget{educaciuxf3n}{%
\section{Educación}\label{educaciuxf3n}}

\begin{cventries}
    \cventry{Doctorado en Ciencias Bioquímicas}{Centro de Ciencias Genómicas, UNAM}{Cuernavaca, Morelos, MX}{2019 - 2023}{}\vspace{-4.0mm}
    \cventry{M. en C. Bioquímicas}{Instituto de Biotecnología, UNAM}{Cuernavaca, Morelos, MX}{2016-2018}{}\vspace{-4.0mm}
    \cventry{Ing. en Biotecnología}{Universidad Politécnica de Pachuca}{Zempoala, Hidalgo, MX}{2010-2014}{}\vspace{-4.0mm}
    \cventry{Lic. en Educación Preescolar}{EGAL-EPRE, CENEVAL}{Mexico City, MX}{2012-2012}{}\vspace{-4.0mm}
    \cventry{Lic. en Educación Media Superior}{Normal Superior Pública del Estado de Hidalgo}{Pachuca, Hidalgo, MX}{2007-2010}{}\vspace{-4.0mm}
\end{cventries}

\hypertarget{investigaciuxf3n}{%
\section{Investigación}\label{investigaciuxf3n}}

\begin{cventries}
    \cventry{Programa de Ecología Genómica  (Centro de Ciencias Genómicas, UNAM)
 \newline Tutora: Dra. Esperanza Martínez Romero
}{Inferencia funcional de microsimbiontes asociados a plantas y animales, simbiosis de Leguminosas-\textit{Bradyrhizobium} y \textit{Comadia redtenbacheri} -Microbioma intestinal}{Doctorado en Ciencias Bioquímicas\newline 2019 - 2023}{}{\begin{cvitems}
\item Plantas y animales vivimos en simbiosis con una gran variedad de microorganismos que tienen un extenso papel en el desarrollo, funcionamiento y salud de sus hospederos. Durante el doctorado estudié la simbiosis específica y generalista abordada con dos modelos:
\item \textbf{\textit{Comadia redtenbacheri}‐microbioma intestinal} mediante el análisis metagenómico de larvas de instar temprano y tardío, ensayos enzimáticos y genómica del hospedero. Proponemos que el microbioma coopera activamente en la degradación de azúcares complejos que el insecto no puede metabolizar, además de brindarle aminoácidos y vitaminas.
\item \textbf{Leguminosas-\textit{Bradyrhizobium}} mediante genómica comparativa de \textit{Bradyrhizobium} y ensayos de nodulación-crecimiento. Reportamos cinco genoma-especies que codifican genes relacionados con la nodulación, fijación de nitrógeno y sistemas de secreción tipo 3, estas funciones se encuentran diferenciadas entre cepas aisladas de diferentes Leguminosas. Reportamos 3 nuevos simbiovars que codifican genes relacionados a la especificidad por hospedero e islas simbióticas.
\end{cvitems}}
    \cventry{Laboratorio de Tecnología Enzimática. (Instituto de Biotecnología, UNAM)
 \newline Tutor: Dr. Agustín López Munguía Canales}{Diversidad de \textit{Weissella confusa} en el pozol y su metabolismo de carbohidratos.}{M. en C. Bioquímicas\newline 2016-2018}{}{\begin{cvitems}
\item En este proyecto estudiamos el papel de \textit{Weissella confusa} en la fermentación del \textit{pozol}, mediante ensayos enzimáticos, genómica comparativa y secuenciación de amplicones. Reportamos que \textit{W. confusa} se encuentra a lo largo de la fermentación en baja abundancia relativa, codifica para actividades enzimáticas secundarias en la degradación de azúcares disponibles en el maíz nixtamalizado. Además, utiliza la sacarosa disponible para sintetizar un polímero con potencial prebiótico y propiedades reológicas. Esta proteína se clonó y expresó de forma heteróloga para su purificación y descripción bioquímica.
\end{cvitems}}
    \cventry{Laboratorio de Aprovechamiento Integral de Recursos Bióticos (Universidad Politécnica de Pachuca) \newline Tutoras: Dra. Yuridia Mercado Flores y Dra. Virginia Mandujano González}{Estudio Molecular y Bioinformático del gene \textit{srxl}1 del hongo fitopatógeno de maíz \textit{Sporisorium reilianum}}{Ing. en Biotecnología\newline 2010-2014}{}{\begin{cvitems}
\item Analicé la región promotora del gene \textit{srxl}1, que codifica para una ß-xilanasa del hongo fitopatógeno de maíz, \textit{Sporisorium reilianum}. Se encontraron sitios de unión a factores transcripcionales que regulan el desarrollo mediante nitrógeno. Por lo que se evaluó el crecimiento a diferentes concentraciones de fuente de nitrógeno y se encontró una correlación en el incremento de la actividad xilanolítica. Por otro lado, se hicieron ensayos para la construcción de una mutante nula de dicho gene.
\end{cvitems}}
\end{cventries}

\hypertarget{publicaciones}{%
\section{Publicaciones}\label{publicaciones}}

\footnotesize
\setlength{\leftskip}{0cm}

\textbf{2023}

\setlength{\leftskip}{1cm}

Rodolfo Enrique Ángeles‐Argáiz; \textbf{Diana Hernández‐Oaxaca}; Luis
Fernando Lozano Aguirre‐Beltrán; Christian Quintero‐Corrales;Mauricio
Trujillo‐Roldán; Santiago Castillo‐Ramírez; Roberto Garibay. (2023).
\emph{Microbial Genomics}. MGEN‐D‐23‐00296 (Enviado)

Rosario Ramírez‐Mendoza, Rodolfo Ángeles‐Argáiz, Luis Lozano‐Aguirre
Beltrán, Juan Almaraz, \textbf{Diana Hernandez‐Oaxaca}, Ivette
Ortiz‐Lopez, and Jesus Perez‐Moreno. Whole‐genome sequence of
\emph{Pseudomonas yamanorum} OLsAu1 isolated from the edible
ectomycorrhizal mushroom \emph{Lactarius} sp. section Deliciosi.
MRA00843‐23R1 (2023) (In press)

López‐Sánchez, R., \textbf{Hernández‐Oaxaca D.}, Escobar‐Zepeda A.,
Ramos‐Cerrillo, B, López‐Munguia, A., Segovia‐Forcella, L. ``Analysing
the dynamics of the bacterial community in \emph{pozol}, a Mexican
fermented corn dough.'' \emph{Microbiology} 169.7 (2023): 001355.

\textbf{Hernández-Oaxaca, D.}, Claro-Mendoza, K. L., Rogel, M. A.,
Rosenblueth, Martínez-Romero J and Martínez-Romero, E. (2023). Novel
symbiovars ingae, lysilomae and lysilomaefficiens in bradyrhizobia from
tree-legume nodules. \emph{Systematic and Applied Microbiology}. 46.4
(2023): 126433.

\setlength{\leftskip}{0cm}

\textbf{2022}

\setlength{\leftskip}{1cm}

\textbf{Hernández-Oaxaca, D.}, Claro-Mendoza, K. L., Rogel, M. A.,
Rosenblueth, M., Velasco-Trejo, J. A., Alarcón-Gutiérrez, E., \ldots{}
\& Martínez-Romero, E. (2022). Genomic Diversity of
\emph{Bradyrhizobium} from the Tree Legumes \emph{Inga} and
\emph{Lysiloma} (Caesalpinioideae-Mimosoid Clade).
\emph{Diversity},14(7), 518.

\setlength{\leftskip}{0cm}

\textbf{2021}

\setlength{\leftskip}{1cm}

\textbf{Hernández-Oaxaca}, D., López-Sánchez, R., Lozano, L.,
Wacher-Rodarte, C., Segovia, L., \& López Munguía, A. (2021). Diversity
of \emph{Weissella confusa} in \emph{Pozol} and Its Carbohydrate
Metabolism. \emph{Frontiers in microbiology}, 12, 572.

Martínez‐Romero, E., Aguirre‐Noyola, J. L., Bustamante‐Brito, R.,
González‐Román, P., \textbf{Hernández‐Oaxaca, D.}, Higareda‐Alvear, V.,
\ldots{} \& Servín‐Garcidueñas, L. E. (2021). We and herbivores eat
endophytes. \emph{Microbial Biotechnology}, 14(4), 1282-1299.

\setlength{\leftskip}{0cm}

\hypertarget{muxe9ritos}{%
\section{Méritos}\label{muxe9ritos}}

\footnotesize

\begin{cventries}
    \cventry{INGEBI-CONICET Buenos Aires, Argentina}{Genome-centric metagenomics for bioremediation and resource recovery}{}{2022}{\begin{cvitems}
\item Curso, vuelo, viáticos y hospedaje.
\end{cvitems}}
    \cventry{Congress Center Lyon, France}{10th International Symbiosis Society Congress. Holobionts in the Anthropocene}{}{2022}{\begin{cvitems}
\item Inscripción
\end{cvitems}}
    \cventry{CeNat-CeniBiot San José, Costa Rica}{Actualización en metagenómica bacteriana y viral.}{}{2022}{\begin{cvitems}
\item Curso, vuelo, viáticos y hospedaje.
\end{cvitems}}
\end{cventries}

\hypertarget{mentoruxedas}{%
\section{Mentorías}\label{mentoruxedas}}

\footnotesize
\begin{cventries}
    \cventry{Mayra Campos Ojeda}{Licenciatura en biología y medio ambiente}{Universidad Guizar y Valencia}{2022-actual}{\begin{cvitems}
\item Identificación de actividad pectinolítica en bacterias intestinales del gusano rojo de maguey: Comadia redtenbacheri
\end{cvitems}}
    \cventry{Axel Emmanuel Pérez Acuña}{Licenciatura en biología y medio ambiente}{Universidad Guizar y Valencia}{2022-actual}{\begin{cvitems}
\item Identificación de bacterias celulolíticas aisladas de intestinos de Comadia redtenbacheri
\end{cvitems}}
    \cventry{Karen Lizbeth Claro Mendoza}{Licenciatura en biología}{Facultad de Ciencias, UNAM}{2021-2022}{\begin{cvitems}
\item Predicción de islas genómicas, análisis filogenéticos y genómicos de una cepa de Bradyrhizobium aislada de los nódulos de Lysiloma sp.
\end{cvitems}}
\end{cventries}

\hypertarget{enseuxf1anza}{%
\section{Enseñanza}\label{enseuxf1anza}}

\begin{cventries}
    \cventry{INECOL}{Análisis de Metagenomas}{\textbf{Ayudante}}{Noviembre}{}\vspace{-4.0mm}
    \cventry{Instituto de ciencias del Mar y Limnología, UNAM.}{Hackeando las comunidades Microbianas}{\textbf{Coordinadora}}{Agosto-Nobiembre, 2023}{}\vspace{-4.0mm}
    \cventry{Unidad Universitaria de Secuenciación Masiva y Bioinformática. IBt, UNAM.}{Curso Herramientas Bioinformáticas 2023}{\textbf{Ayudante}}{Junio, 2023}{}\vspace{-4.0mm}
    \cventry{Instituto de ciencias del Mar y Limnología, UNAM.}{Análisis de Metagenomas}{\textbf{Ayudante}}{Enero, 2023}{}\vspace{-4.0mm}
    \cventry{Red Mexicana de Bioinformática.
CDSB
CCG, UNAM.}{Taller Internacional de Análisis Avanzados de Metagenomas}{\textbf{Ayudante}}{Agosto, 2022}{}\vspace{-4.0mm}
    \cventry{Universidad Autónoma de Querétaro, Facultad de Química. (Virtual)}{Taller de Genómica Funcional de Hongos}{\textbf{Instructora}}{Junio, 2022}{}\vspace{-4.0mm}
    \cventry{Unidad Universitaria de Secuenciación Masiva y Bioinformática. IBt, UNAM.}{Curso Integral para el Análisis de datos de Genómica y Transcriptómica 2022}{\textbf{Ayudante}}{Enero, 2022}{}\vspace{-4.0mm}
    \cventry{Posgrado en Ciencias Bioquímicas}{Microbioma y efectos de la microbiota en hospederos}{\textbf{Coordinadora}}{Agosto-Diciembre, 2021}{}\vspace{-4.0mm}
    \cventry{Unidad Universitaria de Secuenciación Masiva y Bioinformática. IBt, UNAM.}{Curso de Herramientas Bioinformáticas para el Análisis de Datos de Secuenciación Masiva}{\textbf{Ayudante}}{Agosto, 2021}{}\vspace{-4.0mm}
    \cventry{Laboratorio de Ecología Genómica. CCG, UNAM.}{Introducción a Linux}{\textbf{Instructora}}{Enero-Mayo, 2021}{}\vspace{-4.0mm}
    \cventry{Software Carpentry, BetterLab}{R for Metagenomics}{\textbf{Ayudante}}{Marzo, 2021}{}\vspace{-4.0mm}
    \cventry{Unidad Universitaria de Secuenciación Masiva y Bioinformática. IBt, UNAM.}{Curso de Herramientas Bioinformáticas para el Análisis de Datos de Secuenciación Masiva}{\textbf{Ayudante}}{Enero, 2021}{}\vspace{-4.0mm}
    \cventry{Unidad Universitaria de Secuenciación Masiva y Bioinformática. IBt, UNAM.}{Herramientas Bioinformáticas para el Análisis de Datos de Secuenciación Masiva}{\textbf{Ayudante}}{Octubre, 2020}{}\vspace{-4.0mm}
    \cventry{Micromics}{Análisis teórico de metagenomas}{\textbf{Ponente}}{Septiembre, 2019}{}\vspace{-4.0mm}
    \cventry{Jardín de Niños Hontoria, SEP. Pachuca, Hidalgo}{Docente de Educación Preescolar}{\textbf{Docente}}{Feb, 2009 - Jul, 2016}{}\vspace{-4.0mm}
    \cventry{Primaria Amado Nervo, SEP. Atotonilco, Hidalgo}{Docente en Primaria Multigrado}{\textbf{Docente}}{2006-2007}{}\vspace{-4.0mm}
\end{cventries}

\hypertarget{difusiuxf3n-y-divulgaciuxf3n}{%
\section{Difusión y Divulgación}\label{difusiuxf3n-y-divulgaciuxf3n}}

\begin{cventries}
    \cventry{\textbf{Viviendo de aguamiel, la historia del gusano rojo de maguey y su microbioma} \newline {Instituto Morelense de Radio y Televisión}}{}{}{Enero, 2023}{}\vspace{-4.0mm}
    \cventry{\textbf{Inferencia funcional del microbioma asociado a Comadia redtenbacheri} \newline {Semana Académica CCG, UNAM}}{}{}{Diciembre, 2022}{}\vspace{-4.0mm}
    \cventry{\textbf{Un paseo por la metagenómica} \newline {Bioquímica y Biología Molecular, México}}{}{}{Septiembre, 2021}{}\vspace{-4.0mm}
    \cventry{\textbf{Co-organización del primer Club Nacional de Microbiomas} \newline {MicrobeMX Journal Club}}{}{}{Enero-Diciembre, 2021}{}\vspace{-4.0mm}
    \cventry{\textbf{Bichos bajo la lupa: el universo dentro de un insecto} \newline {1er Puertas Abiertas CCG, UNAM}}{}{}{Octubre, 2019}{}\vspace{-4.0mm}
    \cventry{\textbf{Probióticos y Prebióticos: Microbiota Feliz, intestinos sanos.} \newline {Puertas abiertas IBt, UNAM}}{}{}{Abril, 2018}{}\vspace{-4.0mm}
\end{cventries}

\hypertarget{congresos}{%
\section{Congresos}\label{congresos}}

\begin{cvhonors}
    \cvhonor{}{\textbf{The gut microbiome of \textit{Comadia redtenbacheri} contributes to the nutrition of its host} \newline 10th International Symbiosis Society Congress, (Congress Center Lyon, France)}{Póster \newline}{2022\newline}
    \cvhonor{}{\textbf{Microbial interaction in the gut \textit{Comadia redtenbacheri}} \newline Programming for Evolutionary Biology Conference, (Freie Universität Berlin (Online))}{Plática \newline}{2021\newline}
    \cvhonor{}{\textbf{Secondary enzymatic activities in the \textit{pozol} fermentation} \newline Congreso de Biotecnología, (Centro de Congresos
Varadero, Cuba)}{Póster \newline}{2017\newline}
    \cvhonor{}{\textbf{Congreso Nacional de Control Biológico} \newline Centro de Convenciones, (Oaxaca)}{Asistencia \newline}{2013\newline}
\end{cvhonors}

\hypertarget{actualizaciuxf3n}{%
\section{Actualización}\label{actualizaciuxf3n}}

\begin{cventries}
    \cventry{\textbf{Diplomado en Ciencia de Datos y Bioinformática} \newline WBDS, Argentina (virtual)}{}{}{Enero - Abril}{}\vspace{-4.0mm}
    \cventry{\textbf{Genome-centric metagenomics for bioremediation and resource recovery 
} \newline INGEBI-CONICET Buenos Aires, Argentina}{}{}{Noviembre, 2022}{}\vspace{-4.0mm}
    \cventry{\textbf{10th International Symbiosis Society Congress. Holobionts in the Anthropocene
} \newline Congress Center Lyon, France}{}{}{Agosto, 2022}{}\vspace{-4.0mm}
    \cventry{\textbf{Actualización en metagenómica bacteriana y viral.} \newline CeNat-CeniBiot San José, Costa Rica}{}{}{Julio, 2022}{}\vspace{-4.0mm}
    \cventry{\textbf{Exploring Biological Data with Pandas} \newline WBDS, Argentina (virtual)}{}{}{Septiembre, 2021}{}\vspace{-4.0mm}
    \cventry{\textbf{Desarrollo de Flujos en R} \newline Encuentro de Bioinformática en México 2021, NNB y CDSB}{}{}{Agosto, 2021}{}\vspace{-4.0mm}
    \cventry{\textbf{Análisis de Amplicones del gene rDNA 16S} \newline ATGenomics}{}{}{Agosto, 2020}{}\vspace{-4.0mm}
    \cventry{\textbf{Supercomputo aplicado al análisis masivo de metagenomas y genómica comparada} \newline SCAYLE
León, España}{}{}{Octubre, 2019}{}\vspace{-4.0mm}
    \cventry{\textbf{Análisis exploratorio de datos biológicos usando R} \newline Talleres Internacionales de Bioinformática
Nodo Nacional de Bioinformática y CDSB}{}{}{Julio, 2018}{}\vspace{-4.0mm}
    \cventry{\textbf{Cromatorgrafía Líquida de Alta Resolución (HPLC)} \newline Centro de Investigación Científica de Yucatán (CICY)}{}{}{Junio, 2018}{}\vspace{-4.0mm}
    \cventry{\textbf{Ensamble y Anotación de Genomas} \newline Talleres Internacionales de Bioinformática
Nodo Nacional de Bioinformática}{}{}{Julio, 2017}{}\vspace{-4.0mm}
    \cventry{\textbf{Introducción a la Bioinformática} \newline Talleres Internacionales de Bioinformática
Nodo Nacional de Bioinformática}{}{}{Enero, 2017}{}\vspace{-4.0mm}
    \cventry{\textbf{Herramientas Bioinformáticas aplicadas al diseño y análisis de ADN recombinante y expresión de proteínas} \newline Global Agronomics}{}{}{Diciembre, 2013}{}\vspace{-4.0mm}
    \cventry{\textbf{Curso Nacional de Control Biológico} \newline Congreso Nacional de control Biológico}{}{}{Noviembre, 2013}{}\vspace{-4.0mm}
    \cventry{\textbf{Aislamiento y selección de Rhizobacterias} \newline UPP-INIFAP}{}{}{Septiembre, 2013}{}\vspace{-4.0mm}
\end{cventries}

\hypertarget{habilidades}{%
\section{Habilidades}\label{habilidades}}

\textbf{Laboratorio}

\begin{itemize}
\tightlist
\item[$\boxtimes$]
  Extracción de ADN, PCR y Clonación
\item[$\boxtimes$]
  Expresión Heteróloga
\item[$\boxtimes$]
  Cultivos Microbianos y Fermentaciones
\item[$\boxtimes$]
  Ensayos de Actividad Enzimática
\item[$\boxtimes$]
  Geles de proteína y zimogramas
\item[$\boxtimes$]
  Cromatografía
\end{itemize}

\textbf{Lenguajes de Programación}

\begin{itemize}
\tightlist
\item[$\boxtimes$]
  Bash
\item[$\boxtimes$]
  Python
\item[$\boxtimes$]
  R (Básico)
\end{itemize}

Last updated on November 2023.


\label{LastPage}~
\end{document}
